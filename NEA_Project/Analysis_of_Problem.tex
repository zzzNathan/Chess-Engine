\section{Analysis of problem}
\subsection{Problem Identification}
This paper will detail my process of coding a chess engine in 
Python 3.12. The game of chess can be traced back to some 1,500
years ago to it's origins in India, where it was known as 
'Chaturanga'. Chess is a hard game to master, with most master
level players starting to play at the ages of 7-9 years of age. 
The game not only has a multitude of strategies, openings and 
tactics, but also is very mentally taxing. In the 1984 world 
chess championship Anatoly Karpov reportedly lost over 22lbs 
(roughly 10 kg). The game of chess has a branching factor of 
35-38 moves per position, which is a lot of moves to consider per
position. Luckily we now have much better technology than 1984, 
and computers can now process roughly $~10^9 \: \: O(1)$
operations per second! This plays to the strengths of the 
modern computer, even though it doesn't posses the human 
intuition needed to disregard moves that 'look disadvantageous', 
we can simply check all moves available in a given position then 
evaluate which move is the most optimal, assuming our opponent 
always plays the best response.

% https://en.wikipedia.org/wiki/World_Chess_Championship_1984%E2%80%931985
% https://www.chessprogramming.org/Branching_Factor
% https://en.wikipedia.org/wiki/History_of_chess

\subsection{Stakeholders}
Chess has skyrocketed in popularity these past few years, because of
a collection of reasons. One very clear cause for this spike in interest
was the COVID-19 lockdown. This period of time saw everyone stuck
indoors looking for new hobbies, distractions and ways to pass
the time. Not only were there a multitude of content creators
now learning how to play chess, but there was also a new hit 
series "The Queen's Gambit", which according to netflix a record
62 million households watched. All this interest in chess encouraged 
many to start learning chess for themselves, and to analyse their 
games and learn from their mistakes people would have to 
recieve the help of a chess engine. This project will aid 
those people eagerly wanting to improve their chess ability, 
by allowing them to learn from a much higher rated player without 
paying fees for a coach. The target demographic is any avid chess player
from the ages of 9-40 wanting to improve at chess, but lacking
the money to hire a coach.
% https://www.nytimes.com/2022/06/17/crosswords/chess/chess-is-booming.html

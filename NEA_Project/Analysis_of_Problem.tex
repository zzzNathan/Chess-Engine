\pagestyle{fancy}
\fancyhead[LE,LO]{OCR A Level - Computer Science NEA}
\fancyhead[RE,RO]{Jonathan Kasongo}

\chapter{Analysis of the problem}

\section{Problem Identification}
The game of chess has skyrocketed in terms of popularity
recently, so much so that half of my school now spend their
break times playing each other on \url{chess.com}. Chess 
is a strategy board game with the end goal being to checkmate
the opponent's king. This means that capture of 
the opponent's king is inevitable upon the next move. The 
game also involves \textbf{no} elements of luck and the 
outcome of the game is soley dependent on the actions of the
player. Moreover, the game of chess is known to be very
hard to master with many of the best chess
\textit{Grandmasters} starting training from the
ages of 7-8. The game of chess has an average of 30 moves 
per position. This means that if one wants to think three 
moves ahead of his opponent he must consider ~$27,000$ 
positions in total! This is simply not possible for a 
human, however for a computer this task is something that
could be done in less than 1 second. By leveraging the 
high computational power of modern computers, I aim to 
write a chess engine that is able to beat the average 
human chess player 9 times out of 10.

% https://en.wikipedia.org/wiki/World_Chess_Championship_1984%E2%80%931985
% https://www.chessprogramming.org/Branching_Factor
% https://en.wikipedia.org/wiki/History_of_chess

\section{Stakeholders}
% https://www.nytimes.com/2022/06/17/crosswords/chess/chess-is-booming.html

\pagestyle{fancy}
\fancyhead[LE,LO]{OCR A Level - Computer Science NEA}
\fancyhead[RE,RO]{Jonathan Kasongo}

\chapter{Analysis of the problem}

\section{Problem Identification}
The game of chess has skyrocketed in terms of popularity
recently, so much so that half of my school now spend their
break times playing each other on \url{chess.com}. Chess 
is a strategy board game with the end goal being to checkmate
the opponent's king. \cite{rules} This means that capture of 
the opponent's king is inevitable upon the next move. The 
game also involves \textbf{no} elements of luck and the 
outcome of the game is soley dependent on the actions of the
player. Moreover, the game of chess is known to be very
hard to master with many of the best chess
\textit{Grandmasters} starting training from the
ages of 7-8 \cite{Magnus}. The game of chess has an average
of 35 moves \cite{branch} per position. This means that 
if one wants to think three moves ahead of his opponent
he must consider $42,875$ positions in total! This is
simply not possible for a human, however for a computer
this task is something that could be done in less than
1 second. By leveraging the high computational power
of modern computers, I aim to write a chess engine that
is able to beat an average human chess player 9 times
out of 10.\\

Whilst chess prodigies and Grandmasters dedicate their entire
lifes to improving their chess abilities, using high order 
thinking processes, experience and strategical tactics to play
the best move in a position we may simply use a brute-force 
style of play, in which we consider all legal moves from a 
given position and simply choose the one that gives the most
advantageous position even if our opponent doesn't make any 
mistakes.

% https://en.wikipedia.org/wiki/World_Chess_Championship_1984%E2%80%931985
% https://www.chessprogramming.org/Branching_Factor
\section{Stakeholders}
% https://www.nytimes.com/2022/06/17/crosswords/chess/chess-is-booming.html
One of the students at my school who plays chess regularly is 
John Arco. John Arco is a 17 year old male with a passion for chess.
John has a rating of roughly 1000 ELO but wishes 
to improve to a higher rating and beat all of his 
classmates. John is also very competitive and wishes to 
\textbf{guarentee} that none of his classmates can beat him.
The use of a strong chess engine is one method to ensure
that John Arco always beats his classmates and requires little
to no effort on his part, all he has to do is replicate the 
moves played by his opponent on the engine's board and he 
will simply replicate the computer's moves. 
\footnote{I do realise that this is considered cheating,
however we intend to use this engine completely offline
in unrated friendly games against close friends. 
I do not advocate cheating in any way shape or form.}
Moreover using a chess engine can also be highly educational as
we may learn new ideas or moves from the engine that we may have
never considered previously. Even Magnus Carlsen
has openly said that he has learnt new ideas from chess engines.
\cite{lex} This means the engine is to serve 2 purposes,
the first is ensure that John Arco remains undefeated 
against his classmates, and the second is to improve John Arco's
chess ability by exposing him to new and unique tactics that 
he wouldn't have thought of otherwise. The construction of 
a strong chess engine will be able to solve both problems
effectively, providing both educational benefits and 
competitive benefits also.

\section{Research the problem}
To begin research it is first nescessary to get a higher level understanding
of how a chess engine works. To learn about this topic
I made use of resources like \url{https://www.chessprogramming.org/Main_Page}
and \url{https://www.talkchess.com/forum/index.php}, citations will be 
given accordingly.\\

Any chess engine must be comprised of these 3 fundamental components:
\begin{itemize}
  \item \textit{Legal move generation}
  \item \textit{Evaluation functions}
  \item \textit{Searching algorithms}
\end{itemize}

\subsection*{Bitboards}
To understand the following algorithms it is nescessary to 
have a good understanding on \textbf{\textit{bitboards}}.
If you already understand this concept please skip this 
subsection entirely, otherwise I will provide a brief 
introduction to the idea here. Some helpful resources can be
found here \cite{bitboards}.

\subsection*{Legal move generation}
Legal move generation is the first step, in this component we wish to 
find a way to feed in a position to a computer program and have it 
output to us all of the possible legal moves available in this position.
The study of move generation algorithms in the chess programming world 
is still very new, with one of the newest algorithm being
discovered in 2017 \cite{bm}. The two algorithms I decided to research 
were \textit{Hyperbola quintessence} and \textit{Magic bitboards}.\\

Hyperbola quintessence is a bit manipulation algorithm, that
applies the \texttt{o xor (o-2r)} trick to generate correct
and legal moves for the bishop, rook and queen.
